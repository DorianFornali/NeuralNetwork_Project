\chapter{Conclusion}
Our exploration into machine learning applications for football match prediction led us through crucial decisions in data preparation and the implementation of various algorithms. The choice to cut the data in 1996 proved essential, mitigating challenges related to changing political entities and ensuring consistent nomenclature. Structuring the dataset to accommodate country name changes was a pivotal organizational step, laying the groundwork for meaningful analysis and model implementation.\\


The implementation of a regression random forest showed successes and challenges, with 60\% accuracy, biases favoring the home team, and limitations in handling certain scenarios. Championship predictions illustrated the algorithm's ability to simulate complex football dynamics.\\


The implementation of a regression neural network revealed the model's sensitivity to dataset size, with the simpler random forest outperforming the neural network. Correlation analysis provided insights into feature relevance, contributing to a more informed data preparation approach.\\


In conclusion, the two models present divergent results. The random forest offers consistent but less precise predictions, while the neural network generates more chaotic predictions. The random forest appears better suited to this type of dataset, but the addition of additional datasets, such as team composition or medical conditions, could enhance overall prediction accuracy. This journey underscores the need for a balanced approach between model sophistication, data context, and adaptability to the inherent unpredictability of ever-evolving football.
